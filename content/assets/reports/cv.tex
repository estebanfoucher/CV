\documentclass[11pt,a4paper]{article}

% Encoding and language
\usepackage[utf8]{inputenc}
\usepackage[T1]{fontenc}
\usepackage[english]{babel}

% Fonts and layout
\usepackage{lmodern}
\usepackage[margin=2.2cm]{geometry}
\usepackage{setspace}
\setstretch{1.0}
\usepackage{parskip} % no indent, space between paragraphs

% Links
\usepackage{hyperref}
\hypersetup{
  colorlinks=true,
  urlcolor=black,
  linkcolor=black
}

\begin{document}

% Name at top
{\LARGE\bfseries Esteban Foucher}\par\vspace{0.2cm}

% Contact
\href{mailto:estefoucher@gmail.com}{estefoucher@gmail.com}\\
Engineer, physicist -- \'Ecole Normale Sup\'erieure Ulm (class of 2020)

\vspace{0.4cm}

Engineer and physicist trained at École Normale Supérieure Ulm, I work at the intersection of simulation, data, and applied physics. I am currently a thermo-hydraulic simulation engineer at Jimmy Energy, where I contribute to nuclear reactor conception. In parallel, I develop an embedded computer-vision system for aerodynamic measurements.

\vspace{0.6cm}

% ----- Professional Experience -----
{\large\bfseries Professional Experience}\par\vspace{0.3cm}

\textbf{2025 --- Thermo-hydraulic Simulation Engineer, Jimmy Energy}\\\\
Currently:
\begin{itemize}
  \item Thermo-hydraulic simulations (computational fluid dynamics)
\end{itemize}
Internship:
\begin{itemize}
  \item Support for reactor design engineering and safety
  \item Exploratory simulations of chemical kinetics under irradiation
\end{itemize}
My experience at Jimmy includes an initial internship in 2023--2024, followed by joining the simulations team again in 2025.

\vspace{0.3cm}

\textbf{2024 --- Data \& Performance Engineer, Orient Express Racing Team}\\
\begin{itemize}
  \item Data engineering \& analysis for performance of the AC40 and AC75 racing yachts
  \item Data science: development of a post-processing wind-reconstruction tool and an optimizer for calibration
  \item Collaboration with the mechatronics team for debugging and performance follow-up
\end{itemize}
My time at OERT covers in particular the phase of commissioning of the AC75 and participation in the 37th America's Cup.

\vspace{0.3cm}

\textbf{2023 --- M2 Research Internship, Nano-optics}\\\\
Laboratory of Physics, \'Ecole Normale Sup\'erieure -- Nano-optics group\\
\textbf{Topic:} Ultra‑broadband photodetection of the two‑dimensional semi‑metal \textit{PtSe2}.\\
\begin{itemize}
  \item Conception of the ultrabroadband microphotodetection experiment
  \item Fabrication of PtSe2 field-effect transistors
\end{itemize}

\vspace{0.3cm}

\textbf{2022 --- M1 Research Internship, Condensed Matter}\\\\
Theory \& Simulation of Condensed Matter Group -- King's College London\\
\textbf{Topic :} Fully bold formalism at strong coupling regime for the (0 + 0)-dimensionnal Hubbard model (Theoretical study of a strongly interacting electron system)

\vspace{0.3cm}

\textbf{2021 --- L3 Research Internship, Nanofluidics}\\\\
Laboratory of Physics, ENS -- Nanofluidics group\\
\textbf{Topic:} Measurement of fluid flows at the nanoscale using confocal fluorescence microscopy.

\vspace{0.8cm}

% ----- Education -----
{\large\bfseries Education}\par\vspace{0.3cm}

\textbf{2020--2024 --- \'Ecole Normale Sup\'erieure, Paris}\\\\
Master ICFP: Fundamental Physics (quantum physics major)\\
Bachelor's degree: Fundamental Physics

\textit{Main subjects:}
\begin{itemize}
  \item \textbf{Physics:} quantum physics, statistical physics, solid-state physics, special relativity,
        quantum field theory, general relativity, fluid mechanics
  \item \textbf{Computer science / numerical methods:} numerical methods for partial differential equations, machine learning
  \item \textbf{Mathematics:} statistics, probability, optimization, algebra, analysis
\end{itemize}

\vspace{0.8cm}

% ----- Projects -----
{\large\bfseries Projects}\par\vspace{0.3cm}

\textbf{SailCV -- Embedded computer-vision for aerodynamic measurements}\\\\
Motivated by my experience in the sailing world, I am developing in my spare time an embedded
computer-vision project dedicated to aerodynamic measurements.

\textit{SailCV-tell-tale-tracker}\\
Design of a tell-tale tracker to monitor boundary-layer separation. Creation of a dataset and
fine-tuning of a detector.

\textit{SailCV-3D-reconstruction}\\
This project aims to accurately reconstruct metric point clouds from calibrated stereo views. It is based on:
\begin{itemize}
  \item the ability of recent AI models dedicated to 3D reconstruction to predict dense point correspondences between two views of the same object;
  \item precise calibration of the intrinsic and extrinsic parameters of a dual-camera system,
        enabling accurate triangulation.
\end{itemize}

\vspace{0.3cm}

\textbf{Normale Physics Review}\\\\
Contribution to the creation and writing of the student physics journal
\href{https://normalephysicsreview.netlify.app/}{Normale Physics Review}.

\end{document}


